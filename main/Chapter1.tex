\chapter{Meshless methods}\label{chap:lorem}

Meshless or Meshfree methods (MM) were developed to overcome the drawbacks of traditional mesh-based methods.
They appeared for the first time in $1977$ with the Smooth Particle Hydrodynamics method~\cite{Belytschko:meshless_overview}, initially used to modeling astrophysical phenomena such as exploding stars and dust clouds. The same method was later applied in solid mechanics to overcome limitations of mesh-based methods; their advantage is that \textit{the approximation of unknowns in the PDE is constructed based on scattered points without mesh connectivity}~\cite{Chen:meshless_overview_after_20_years}.

Basic principle of MM is the construction of an approximating field $u^{h}$ for the sought solution $u $ of the PDE, expressed by the following expansion:
\begin{equation}
	\label{eqn:general_u_discretization}
	u^{h}(\vec{x}) = \sum_{k= 1}^{N} {\alpha_k B_k(\vec{x})}
\end{equation}
where $\vec{x} \in \R^{d}$ is one of the N generated nodes $\Set{\vec{x}_1 \dots \vec{x}_N}$ distributed over the physical domain $\Omega \subset \R^{d}$, $B_k(\vec{x})$ are the basis functions and $\alpha_k$ are the expansions coefficients.

Different choices for the basis functions $B_k$ leads to different formulations and in literature can be found several of these, some examples are:  reproducing kernel particle method (RKPM)~\cite{Liu:RKPM}, moving least square (MLS)~\cite{Lancaster:MLS}, radial basis function (RBF)~\cite{Kansa:RBF_1},~\cite{Kansa:RBF_2} and partition of unity (PU)~\cite{Schweitzer:PU}.

Now that we defined the general form of our approximated solution for the PDE, we can employ it for the discretization of the PDE.
First consider a general PDE on $\Omega$ with a boundary $\partial\Omega=\Gamma$:
\begin{equation}
	\label{sys:generic_continous_PDE}
	\begin{cases}
		\mathcal{L} u(\vec{x})  = f(\vec{x})		& \text{in $\Omega$} \\
		u(\vec{x}) 							   = g(\vec{x})	 & \text{on $\Gamma$}
	\end{cases}
\end{equation}
where $\mathcal{L}$ is a linear differential operator, $f$ and $g$ are known function. If  we now replace our approximation ~\eqref{eqn:general_u_discretization} in~\eqref{sys:generic_continous_PDE} we will obtain, in general, a non-zero error function $\epsilon^{h}$ given by:
\begin{equation}
	\epsilon^{h}(\vec{x}) = \mathcal{L} u^{h}(\vec{x}) - f(\vec{x})
\end{equation}
A set of test function $\Upsilon$ orthogonal to $\epsilon^{h}$ are then used to integrate the error to zero:
\begin{equation}
	\begin{split}
		\int_{\Omega} \Upsilon \epsilon^{h}\,d\Omega & = \int_{\Omega} \Upsilon (\mathcal{L} u^{h}(\vec{x}) - f(\vec{x}))\,d\Omega  \\
		& = \int_{\Omega} \Upsilon \Bigl[ \mathcal{L}\Bigl( \sum_{k= 1}^{N} {\alpha_k B_k(\vec{x})}\Bigr) - f(\vec{x})\Bigr]\,d\Omega = 0
	\end{split}
\end{equation}
and different choices of $\Upsilon$ leads to different formulations~\cite{Chen:meshless_overview_after_20_years}:
\begin{description}
	\item[Galerking Meshless Methods] that use weak form of PDE. These require domain integration and require special techniques to enforce boundary conditions;
	\item[Collocation Meshless Methods] that use strong form of PDE and allows to solve them directly on the generated nodes. Further they do not require domain integration nor special procedures to deal with boundary conditions
\end{description}

\section{Ipsum}

Citation sample~\cite{latexcompanion}

\begin{figure}[!ht]\label{fig:example}
\centering
\includegraphics[width=.8\textwidth]{img/units_logo.png}
\caption{A figure}
\end{figure}

\lipsum[3]

\begin{figure}[!ht]
\centering
\subfigure[Subfigure 1]{\includegraphics[width=.45\textwidth]{img/units_logo.png}}
\subfigure[Subfigure 2]{\includegraphics[width=.45\textwidth]{img/units_logo.png}}
\subfigure[Subfigure 3]{\includegraphics[width=.45\textwidth]{img/units_logo.png}}
\subfigure[Subfigure 4]{\includegraphics[width=.45\textwidth]{img/units_logo.png}}
\caption{A figure with subfigures}
\label{fig:subexample}
\end{figure}

\lipsum[4]

\section{Dolor}
\lipsum[3]

\begin{table}
\centering
\begin{tabular}{|c|c|}
\hline
\textbf{Lorem} & \textbf{Ipsum} \\
\hline
\hline
Dolor & Sit \\
\hline
Amet &  \\
\hline
\end{tabular}
\caption{A table}
\label{tab:example}
\end{table}

\lipsum[4-5]