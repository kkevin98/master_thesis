\chapter*{Conclusion}
% --------- Do not edit ---------
\addcontentsline{toc}{chapter}{Conclusion}
\label{chap:concl}
\markboth{CONCLUSION}{CONCLUSION}
%--------------------------------

The present study aims to investigate design optimization problems based on physical models whose behavior can be described by means of Poisson equation.

For the solution of the Poisson equations we decided to leverage on meshless methods, in particular RBF-FD and RBF-HFD methods, which are quickest approaches respect the standard Finite Element Methods (FEMs) or Finite Volume Methods (FVMs) and, even more importantly, does not ask the creation of a mesh which require a well trained operator in order to be created.

We addressed the possibility to introduce:
\begin{itemize}
	\item Automatic Differentiation (AD), a family of techniques for evaluating derivatives of numeric functions expressed as computer programs which enable efficient and accurate computations;
	\item adjoint method, a well established technique used in Computational Fluid Dynamics (CFD) design optimization based on FEMs and FVMs.
\end{itemize}
in simple $1$D and $3$D problems defined by us.
After an introduction to the RBF-FD and RBF-HFD methods we analyzed the structure of design optimization problems. In particular we have shown how AD and adjoint method can be used for gradient-based optimization algorithms which require the computation of the gradient of the cost function.

The mathematical steps shown are then subsequently implemented using the \verb*|Julia| programming language where we have used the \verb*|Zygote| library to implement the part related to the automatic differentiation. For the adjoint method, instead, the code has been written completely by us without using any pre-existent package.

The implemented algorithms are then initially tested on a simple $1$D problem to test their correctness leading to the correct minimization of the defined objective function.
After this case we also faced a $3$D optimization problem which, although simple, allowed for a physical interpretation related to thermal CFD. In this case we shown the correctness of our approach which has been compared with the continuous adjoint providing the same results.

\medskip
Beyond the good results achieved with the optimization problems faced on this work, further studies can be conducted regarding the topics discussed in this thesis:
\begin{itemize}
	\item test AD and adjoint method based on more complex models different from Poisson equations with more complex $3$D domains;
	\item generalizing the proposed approach to more general objective functions in the $3$D case;
	\item modify the implemented code so that it can be executed using multiple threads at once, thus reducing the computational time;
	\item explore the possibility of using automatic differentiation, through the functions provided by \verb*|Zygote|, directly for the calculation of the objective function gradient.
\end{itemize}

